\chapter{HTML5}
\label{chapter:html5}

HTML5 is a cross-platform and device form-factor agnostic markup
language for defining structured documents. It is a backward
compatible revision of older HTML standards bringing lots of new
functionality, removing unneeded features, and officially documenting
some ``de facto'' standards already supported by some or several web
browsers. \cite{pilgrim2010html5}

In the early 2000s, \abbr{W3C} was developing \abbr{XHTML} and
\abbr{XForms} standards to be the future of the Web. Many parts of
these standards were backward incompatible and required very strict
and error-free authoring. Being frustrated with this vision that was
seen as impractical for the real world, a group of web browser vendors
and other interested parties had a competing vision of the future of
the Web: evolving HTML4 to include additional features maintaining
backward compatibility. W3C members did not agree with this vision,
and as a result, the WHAT Working Group was born. \abbr{WHATWG}
defines itself as a ``loose, unofficial, and open collaboration of Web
browser manufacturers and interested
parties''\footnote{\url{http://www.whatwg.org/news/start}}. \cite{pilgrim2010html5}

According to a
study\footnote{\url{http://dev.opera.com/articles/view/mama-key-findings/}}
made by Opera in 2008, more than 95\% of web sites do not pass markup
validation. Therefore, to maintain backward compatibility and
practicality, it is crucial to have a well defined error handling
mechanism.

Having the browser vendors and web development community support
behind them, after several years the WHATWG work was finally accepted
by W3C and a joint effort was started to standardize HTML5. There are
still differences in the W3C and WHATWG specifications in what
features they include in the main standard and what are separated in
other specifications or leaved out, but the main goal is to develop
the standards together with browser vendors to get usage feedback
while the specifications are being made. This results in many features
being available in modern web browsers while the HTML5 and related
standards are not yet finished. As a drawback, however, the
implementations might change between browser versions, and developers
must take extra effort in detecting the supported
features. \cite{pilgrim2010html5}

In this work, I look at HTML5 beyond the main specifications, and take
into account also related standards that affect modern web application
development. Also, the differences between the W3C and the WHATWG
specifications are not separated since they are not clear-cut. This is
the practical view that, in my opinion, the web development community
has on HTML5.

\section{Semantic Markup}

Google did a study\footnote{\url{http://code.google.com/webstats/}} in
2005 of a sample of over a billion HTML documents about the popular
class names, elements, attributes and related metadata. This analysis
had a large impact on which elements and attributes were considered in
the upcoming HTML5 standard.

HTML5 defines several new elements and attributes. The objective is to
make the markup more semantic for developers and for content
processors such as search engines and screen readers.

The specification aims for a more semantic structure of HTML by
dropping many presentational features. The rationale behind this is
explained with the following reasons \cite{HTML5draft}:

\begin{itemize}
\item Media-independent markup works for more users and yields better
  accessibility
\item Having style-independent markup separates document structure
  from its layout and makes maintenance easier
\item Separating styling results in smaller document sizes.
\end{itemize}

\noindent Each element in HTML5 is in zero or more content categories
that group elements with similar characteristics \cite{HTML5draft}:

\begin{itemize}
\item \textbf{Metadata content:} Content that sets the behavior of the
  document, sets its relationships to other documents, or conveys
  other information of the document.

  \textit{Examples:} \texttt{link}, \texttt{meta}, \texttt{script},
  \texttt{title}

\item \textbf{Flow content:} Most content that is used in the body of
  a document.

  \textit{Examples:} \texttt{a}, \texttt{article}, \texttt{audio},
  \texttt{div}, \texttt{header}, \texttt{form}, \texttt{nav},
  \texttt{p}

\item \textbf{Sectioning content:} Content that defines the scope of
  headings and footers.

  \textit{Examples:} \texttt{article}, \texttt{aside}, \texttt{nav},
  \texttt{section}

\item \textbf{Heading content:} Content that defines a header of a
  section.

  \textit{Examples:} \texttt{h1}, \texttt{h2}, \texttt{hgroup}

\item \textbf{Phrasing content:} Content that holds or marks up the
  text of the document.

  \textit{Examples:} \texttt{abbr}, \texttt{audio}, \texttt{canvas},
  \texttt{img}, \texttt{em}

\item \textbf{Embedded content:} Content that imports another resource
  or inserts content from another vocabulary into the document.

  \textit{Examples:} \texttt{audio}, \texttt{embed}, \texttt{iframe},
  \texttt{img}

\item \textbf{Interactive content:} Content that is intended for user
  interaction.

  \textit{Examples:} \texttt{a}, \texttt{button}, \texttt{menu},
  \texttt{select}

\end{itemize}

\section{Extensibility}

HTML5 defines the main constructs of a semantic and accessible
document. However, some specific use cases require a more precise and
context-dependent and fine-grained semantics. Also, web browsers might
introduce new features that must conform to the standards. This is why
HTML5 is made extensible for adding more semantics or additional
features on top of the existing standard.

There are several ways to extend HTML5. The simplest approaches
include using the defined general attributes with certain
vocabularies. For example,
microformats\footnote{\url{http://microformats.org/}} and
Schema.org\footnote{\url{http://schema.org/}} define common elements
and class names with certain semantics for defining document metadata.

HTML5 also defines explicit mechanisms for extending the markup
structure. Using \texttt{data-*=""} and \texttt{rel} attributes,
\texttt{meta} tags, or a generic microdata mechanism, the semantics of
the content can be enhanced for automatic reasoning and machine
readability. \cite{HTML5draft}

\section{Media}

Multimedia support is crucial for modern applications. HTML5 defines
elements and APIs for audio, video, subtitles, and embedded content.

Previously to use these rich content types, developers have had to
rely on third-party plugins and browser extensions. Not having to rely
on plugins and extensions has been one of the main goals of the HTML5
standard for improving the openness and accessibility of web content.

\section{Canvas 2D Context}

HTML5 defines the \texttt{canvas} element. It is a
resolution-dependent bitmap canvas for dynamically rendering
graphics. It can be used, for example, for graphs, games, or other
visuals. \cite{HTML5draft}

The Canvas 2D Context specification draft \cite{canvas2Ddraft} defines
a JavaScript API for programmatically drawing on the 2D canvas
surface. The API defines functions for drawing shapes, paths, text,
gradients, and images on the canvas and other functions for handling
the bitmap data.

\section{Form Enhancements}

Forms are an essential construction in interactive HTML
documents. However, due to their relative simplicity in terms of
expressiveness and the lack of proper accessibility features,
developers have been forced to build lots of JavaScript solutions to
enhance and fix some of these problems.

HTML5 brings several enhancements to forms. New input types for
numbers, dates, email addresses, etc. obsolete the need of scripted
widgets by using native platform controls. New form attributes like
\texttt{placeholder} and \texttt{autofocus} bring easy-to-use
accessibility and usability improvements and also reduce the need for
scripting. \cite{HTML5draft}

These additions and enhancements work especially well in mobile
context where user input is slow and cumbersome. For example, by
having a numeric input field lets the mobile platform open the numeric
keyboard by default, which greatly improves the usability of
forms. Automatic form validation in the client side also reduces the
need for unneeded page refreshes since the browser can show error
messages in invalid fields without any JavaScript validation.

\section{Session History Manipulation}

HTML was originally designed to be based on documents and hyperlinks
between these distinct documents with each of them having a unique
\abbr{URL}. This hyperlinked structure, however, does not suit well
for web applications with dynamic content and interactively changing
user interface.

Two of the basic functionalities that users are accustomed to are
bookmarking and going back and forth in the session
history. Traditionally these have been compromised in dynamic
\abbr{Ajax} applications or handled with a lot of extra work.

HTML5 addresses these issues by allowing the developers dynamically
manipulate the session history. The history stack can be changed and
used for navigation and even the browser address bar can be changed
without extra page refreshes. \cite{HTML5draft}

\section{Offline Web Applications}

By design, web sites have always needed a working network
connection. Applications, however, should be able to work offline or
in unreliable and flaky networks. Especially mobile networks are
unreliable \cite{zandy2002reliable}, which has raised the need for
offline support in HTML5.

There are several ways to enable offline support in HTML5
applications. I present these approaches in the following sections.

\subsection{Application Cache}
\label{section:appcache}

\abbr{AppCache} is a relatively simple way to indicate all resources
needed for offline functionality. A manifest file is defined in the
HTML document, and within the file there are sections for resources
that should always or never be cached as well as fallback \abbr{URLs}
for resources that are not cached but with which the fetching
fails. In addition to the simple manifest file listing offline
resources, JavaScript events are defined for cache
changes. \cite{HTML5draft} \\

\clearpage
\noindent \textbf{Example manifest file:}
\begin{verbatim}
CACHE MANIFEST

# Example manifest version 1.

# The resources in this section are cached for offline use.
CACHE:
js/scripts.js
css/styles.css
img/sprite.png
http://example.org/external-image.jpg

# The resources in this section require the user to be online.
NETWORK:
/login

# This section defines resources and their fallback
# URLs if they are inaccessible.
FALLBACK:
/ /offline.html
\end{verbatim}

\subsection{Data Storage}
\label{section:datastorage}

Storing data in the client side has traditionally been constrained
into using cookies, but HTML5 specifies new options for data
persistence.

Two different key/value storages are defined: \texttt{localStorage}
and\\ \texttt{sessionStorage}. The \abbr{API} is same with both of
these, but with\\ \texttt{sessionStorage}, the data is persisted only
for the current browser session. These interfaces are very simple and
easy to use, but are constrained into storing only textual
data. \cite{webstoragedraft}

Two more expressive storage \abbr{APIs} have been specified: client
side \abbr{SQL} database \cite{webstoragedraft} and the Indexed
Database \cite{indexedDBdraft}. The client side SQL database defines
an asynchronous and transactional JavaScript API for a SQL
database. Although being very expressive, due to the relative
complexity compared to simple and scalable key/value storage options,
it is yet to be seen if the client side SQL storage will be accepted
by the browser vendors and developers.

Indexed Database provides synchronous and asynchronous APIs for
storing and querying large amounts of structured data. The
transactional API can be used for more complex persistency needs than
with the simple key/value storages, and it provides a native
JavaScript API that does not involve the relative complexities of SQL.

\subsection{Detecting Network State}

Knowing whether the user is online or offline can affect the user
interface or the response to user interactions. HTML5 defines
functionality to detect the current network status and events that are
fired when the status changes. \cite{HTML5draft}

Although providing important information, these network status
indicators are inherently unreliable \cite{HTML5draft}. Due to the
distributed ad-hoc architecture of the network and possible local or
external proxies or middleware, the application can never be sure if
the network is connected or not. The only option is just to attempt to
make requests and wait for the response or possible failure.

Therefore, applications should be designed to expect the network to
work, but to degrade gracefully when the connection is lost or seems
to be unreliable.

\section{Drag and Drop}
\label{section:dragdrop}

Drag and Drop is a common interaction technique where elements can be
moved within the user interface from one place to another. Older
browsers have had proprietary solutions for this interaction pattern,
but HTML5 standardizes the API.

The specification defines the element attributes and \abbr{DOM} events
for easily enabling and controlling draggable elements and drop
targets. Custom cross-browser JavaScript solutions have enabled this
interaction before, but little JavaScript code is needed with the new
API. The browser handles the interaction and the dynamic rendering,
reducing the interface lagging and the need for extra processing.

\section{SVG and MathML}

While not part of the HTML5 standard, the specification allows for
embedding \abbr{SVG} \cite{SVGTiny12} and \abbr{MathML} \cite{MathML}
markup within HTML.

SVG is a markup language for describing two-dimensional vector
graphics in \abbr{XML}. The markup can be accessed with the \abbr{DOM}
API for creating dynamic and interactive functionality. Within an
HTML5 document, SVG markup can be embedded within the \texttt{svg}
element.

MathML is an \abbr{XML} markup language for describing the structure
and content of mathematical notation. It can be embedded within an
HTML5 document with the \texttt{math} element.

Both of these languages reduce the need of custom images, making the
content more accessible, dynamic, and enabling dynamic interaction
with it. Also, vector graphics can be scaled to fit the available
space no matter what the screen size is, which improves the
cross-platform usefulness for different device form factors.
