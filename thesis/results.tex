\chapter{Results: What Was Good and Where Were the Compromises}
\label{chapter:results}

\section{Targeting Different Platforms}
\label{section:targeting-platforms}

Despite the web browser being the unified environment for different
platforms, there are lots of differences between various devices. The
form factors vary from tiny mobile screens to touch screen tablets and
desktop monitors and each device and platform has its own feature
set. There are also known bugs in the browsers that have to be
handled.

Therefore, means to detect the user's device are needed. Here we
present two such means: device detection and feature detection. Both
of these were used in our conference application.

\subsection{Device Detection}

The User-Agent (\abbr{UA}) \abbr{HTTP} header contains detailed
information of the web browser and platform where the request
originates. As we can see from Table~\ref{table:user-agents}
(\fixme{Check table ref number}), we can extract platform and browser
specific information from the UA header.

\begin{table}
  \begin{tabular}{ l | l | p{7cm} }
    \textbf{Device} & \textbf{Platform} & \textbf{User-Agent} \\ \hline
    Samsung Nexus S & Android 2.3.4 & Mozilla/5.0 (Linux; U; Android 2.3.4; en-us; Nexus S Build/GRJ22) AppleWebKit/533.1 (KHTML, like Gecko) Version/4.0 Mobile Safari/533.1 \\ \hline
    Apple iPhone & iOS 3.1.3 & Mozilla/5.0 (iPhone; U; CPU iPhone OS 3\_1\_3 like Mac OS X; de-de) AppleWebKit/528.18 (KHTML, like Gecko) Version/4.0 Mobile/7E18 Safari/528.16 \\ \hline
    Apple iPad & iOS 5.0 & Mozilla/5.0 (iPad; CPU OS 5\_0 like Mac OS X) AppleWebKit/534.46 (KHTML, like Gecko) Mobile/9A334 \\ \hline
    Unknown & Android & Opera/9.80 (Android; Opera Mini/6.5.26571/26.1023; U; de) Presto/2.8.119 Version/10.54 \\ \hline
  \end{tabular}
  \label{table:user-agents}
  \caption{Example User-Agent strings.}
\end{table}

In the conference application, device detection was used in the
backend to provide a different offline AppCache manifest to different
device groups. The detection was also used in defining the assets to
be preloaded in the application. The devices were divided into four
categories based on the rules defined in
Table~\ref{table:device-detection-rules} (\fixme{Check table ref
  number}). There were serious limitations in this approach, and
compromises had to be made.

First, there is no way to surely know if the device actually is what
it reports itself to be. Second, the most important thing to know when
generating the screen specific assets in the manifest file would have
been the screen size. However, this information is not present in the
UA header. We could have listed all the assets for all the devices,
but then the list of offline assets would have grown too much and, for
example, have large images also for older mobile phones.

Despite the drawbacks, the received advantages of this approach
outweighed the possible compromises. The worst that could happen was
that the device was wrongly classified and the proper resources were
not downloaded for offline use.

\begin{table}
  \begin{tabular}{ l | l }
    \textbf{Rule} & \textbf{Device Type} \\ \hline
    'iPad' in UA & highres \\
    'iPhone' in UA & iphone \\
    'Android 3' in UA & highres \\
    'mobile' (case insensitive) in UA & mobile \\
    'MIDP' in UA & mobile \\
    'Opera Mobi' in UA & mobile \\
    'Opera Mini' in UA & mobile \\
    otherwise (desktop computer) & highres
  \end{tabular}
  \label{table:device-detection-rules}
  \caption{Device type detection rules.}
\end{table}

Getting platform and browser information from the UA header might look
tempting and useful, but it is considered a bad practice to detect a
device from it and provide device specific bug fixes or additional
features. The header can easily be changed and some browsers or
browser plugins even provide preconfigured values for certain browsers
or devices for spoofing. Also, the device specific bug fixes might
become obsolete with platform updates, and the application might break
due to invalid expectations. This is why feature detection is
generally the recommended option whenever possible.

\subsection{Feature Detection}

Feature detection is an important concept in Progressive Enhancement
design (See Section~\ref{subsection:progressive-enhancement}). A lot
of the HTML5 related JavaScript APIs are still unsupported in several
platforms, but browser developers are constantly filling the
gaps. Therefore, it is important to check whether a certain feature is
supported and provide graceful fallback mechanisms for browsers
lacking the functionality.

Doing runtime feature detection provides the possibility to give
additional functionality to modern browsers and instant support for
devices that add the feature support in the lifetime of the
application. In the conference application (\fixme{ref needed?}), we
used the Modernizr feature detection library (\fixme{already ref
  earlier, add bib entry?}) to check for HTML5 features.

For example, the user could add sessions to his or her favorites by
clicking the star in the agenda or on the session details view
(\fixme{add screenshot?}). The favorites were then listed on the home
view together with information about the time left for them to begin.

We used HTML5 localStorage for storing the favorites in the user's web
browser. By using Modernizr, we detected localStorage support and
showed the favorite stars only in browsers that supported the
functionality. For all other browsers, the stars were simply hidden
and users could not add favorites. We could have also provided a
fallback mechanism for persisting the favorites to the backend, but
for simplicity and because we targeted mostly modern platforms, this
approach was considered as reasonable.

\section{Targeting Different Screens}
\label{section:targeting-screens}

Probably the biggest difference in various devices and form factors is
the screen size, resolution, and dimensions. Web applications should
adjust to the available space and flexible handle screen orientation
and window size changes.

First, to target mobile and tablet platforms, the viewport meta
information should be indicated in the document. The following tag was
used in the conference application:

\begin{verbatim}
<meta name="viewport" content="width=device-width,
                               initial-scale=1.0">
\end{verbatim}

The viewport meta tag was first introduced in Apple's iPhone and
afterwards ported to other platforms, such as Android. The possible
configuration options and default values might vary between
platforms. Values accepted by Android are shown in
Table~\ref{table:viewport-meta} (\fixme{Check table ref number})
\citationneeded. iOS devices also support these same properties.

\begin{table}
  \begin{tabular}{ l | l | p{5cm} }
    \textbf{Property} & \textbf{Description} & \textbf{Value} \\ \hline
    height & Height of the viewport. & pixel value or 'device-height' \\
    width & Width of the viewport. & pixel value or 'device-width' \\
    initial-scale & Initial zoom level. & float value (0.01--10) \\
    minimum-scale & Minimum zoom level. & float value (0.01--10) \\
    maximum-scale & Maximum zoom level. & float value (0.01--10) \\
    user-scalable & Enables/disables zoom. & 'yes' or 'no' \\
    target-densitydpi & Visual pixel density. & dpi value, 'device-dpi', 'high-dpi', 'medium-dpi', or 'low-dpi' \\
  \end{tabular}
  \label{table:viewport-meta}
  \caption{Viewport meta tag configuration for Android.}
\end{table}

If we do not set the viewport configuration tag, the device uses its
own default values for the properties. For example, the default value
for the width property is 980 pixels in iOS \citationneeded, which is
clearly defined for web sites targeting desktop browsers. Without
setting this value to something smaller and more appropriate in a
mobile context, the whole application is very wide and has small and
unreadable text in the initial zoom level.

In the viewport configuration we used for the conference application
(as defined above), we set the viewport width to 'device\_width'. This
makes the application width to adjust to the visual pixels of the
device screen and works well with screens of different sizes and
dimensions. The only other viewport property we set is the initial
scaling. This is set to 1.0 to force the browser to render the
application without any initial zooming.

In addition to the viewport configuration, we used media queries
\citationneeded to use better background images for high resolution
screens. We also dynamically set the map view (\fixme{Add
  screenshot?}) images based on the screen dimensions so that we could
provide smaller images for smaller screens and high resolution images
for tablets and other devices with larger screen estate.

\section{Handling Different Orientations}

As shown in the previous section, screen sizes and dimensions vary
between devices. In addition to handling different resolutions and
dimensions, we must also handle screen orientation changes. The width
and height of the touch screens are usually different, and the user
can hold the device either in portrait or in landscape mode and in any
point switch between these two.

In the conference application, we wanted to have different header and
footer background images for different orientations. We also needed to
redraw the agenda view when the screen width changes since the items
on the schedule needed to be dynamically positioned to the available
space.

Mobile browsers trigger an 'orientationchange' event whenever the
device orientation changes. We listened to this event, inferred the
orientation from the screen dimensions, and executed the wanted
functionality for the event. We also had to do a fallback for Mobile
\abbr{IE} browser to listen to the window resize event because the
browser does not support the orientation change event.

\section{Handling Mobile Networks}
\label{section:handling-networks}

\subsection{Minimizing Data Transfer}
\subsection{Caching}
\subsection{Preloading}
\subsection{Offline Support}
\subsection{Handling Interruptions}

\section{Graphics and Animations}
\label{section:graphics}

\section{Following JavaScript Best Practices}
\label{section:js-best-practices}

\subsection{JSLint}
\subsection{Lazy initialization}
\subsection{Efficient DOM Manipulation}
\subsection{Efficient Event Handling}

\section{Performance Analysis}
\label{section:performance-analysis}

\subsection{YSlow}
\subsection{PageSpeed}
