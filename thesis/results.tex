\chapter{Results: What Was Good and Where Were the Compromises}
\label{chapter:results}

\section{Targeting Different Platforms}
\label{section:targeting-platforms}

Despite the web browser being the unified environment for different
platforms, there are lots of differences between various devices. The
form factors vary from tiny mobile screens to touch screen tablets and
desktop monitors and each device and platform has its own feature
set. There are also known bugs in the browsers that have to be
handled.

Therefore, means to detect the user's device are needed. Here we
present two such means: device detection and feature detection. Both
of these were used in our conference application.

\subsection{Device Detection}

The User-Agent (\abbr{UA}) \abbr{HTTP} header contains detailed
information of the web browser and platform where the request
originates. As we can see from Table~\ref{table:user-agents}
(\fixme{Check table ref number}), we can extract platform and browser
specific information from the UA header.

\begin{table}
  \begin{tabular}{ l | l | p{7cm} }
    \textbf{Device} & \textbf{Platform} & \textbf{User-Agent} \\ \hline
    Samsung Nexus S & Android 2.3.4 & Mozilla/5.0 (Linux; U; Android 2.3.4; en-us; Nexus S Build/GRJ22) AppleWebKit/533.1 (KHTML, like Gecko) Version/4.0 Mobile Safari/533.1 \\ \hline
    Apple iPhone & iOS 3.1.3 & Mozilla/5.0 (iPhone; U; CPU iPhone OS 3\_1\_3 like Mac OS X; de-de) AppleWebKit/528.18 (KHTML, like Gecko) Version/4.0 Mobile/7E18 Safari/528.16 \\ \hline
    Apple iPad & iOS 5.0 & Mozilla/5.0 (iPad; CPU OS 5\_0 like Mac OS X) AppleWebKit/534.46 (KHTML, like Gecko) Mobile/9A334 \\ \hline
    Unknown & Android & Opera/9.80 (Android; Opera Mini/6.5.26571/26.1023; U; de) Presto/2.8.119 Version/10.54 \\ \hline
  \end{tabular}
  \label{table:user-agents}
  \caption{Example User-Agent strings.}
\end{table}

In the conference application, device detection was used in the
backend to provide a different offline AppCache manifest to different
device groups. The detection was also used in defining the assets to
be preloaded in the application. The devices were divided into four
categories based on the rules defined in
Table~\ref{table:device-detection-rules} (\fixme{Check table ref
  number}). There were serious limitations in this approach, and
compromises had to be made.

First, there is no way to surely know if the device actually is what
it reports itself to be. Second, the most important thing to know when
generating the screen specific assets in the manifest file would have
been the screen size. However, this information is not present in the
UA header. We could have listed all the assets for all the devices,
but then the list of offline assets would have grown too much and, for
example, have large images also for older mobile phones.

Despite the drawbacks, the received advantages of this approach
outweighed the possible compromises. The worst that could happen was
that the device was wrongly classified and the proper resources were
not downloaded for offline use.

\begin{table}
  \begin{tabular}{ l | l }
    \textbf{Rule} & \textbf{Device Type} \\ \hline
    'iPad' in UA & highres \\
    'iPhone' in UA & iphone \\
    'Android 3' in UA & highres \\
    'mobile' (case insensitive) in UA & mobile \\
    'MIDP' in UA & mobile \\
    'Opera Mobi' in UA & mobile \\
    'Opera Mini' in UA & mobile \\
    otherwise (desktop computer) & highres
  \end{tabular}
  \label{table:device-detection-rules}
  \caption{Device type detection rules.}
\end{table}

Getting platform and browser information from the UA header might look
tempting and useful, but it is considered a bad practice to detect a
device from it and provide device specific bug fixes or additional
features. The header can easily be changed and some browsers or
browser plugins even provide preconfigured values for certain browsers
or devices for spoofing. Also, the device specific bug fixes might
become obsolete with platform updates, and the application might break
due to invalid expectations. This is why feature detection is
generally the recommended option whenever possible.

\subsection{Feature Detection}



\section{Targeting Different Screens And Orientations}
\label{section:targeting-screens}



\section{Handling Mobile Networks}
\label{section:handling-networks}

\subsection{Minimizing Data Transfer}
\subsection{Caching}
\subsection{Preloading}
\subsection{Offline Support}
\subsection{Handling Interruptions}

\section{Graphics and Animations}
\label{section:graphics}

\section{Following JavaScript Best Practices}
\label{section:js-best-practices}

\subsection{JSLint}
\subsection{Lazy initialization}
\subsection{Efficient DOM Manipulation}
\subsection{Efficient Event Handling}

\section{Performance Analysis}
\label{section:performance-analysis}

\subsection{YSlow}
\subsection{PageSpeed}
