\chapter{Use Case}
\label{chapter:methods}

The Qt Developer
Days\footnote{\url{http://qt.nokia.com/qtdevdays2011/}} is a
conference for developers using the Qt cross-platform application and
user interface framework\footnote{\url{http://qt.nokia.com/}}. I
created a mobile web application with contextual and personalized
session information and daily schedule for the conference.

In this chapter I present the implementation requirements for the
conference application as well as the requirements for the JSONCache
network utility library.

\section{Conference Application Requirements}

The target group for the conference application was mobile developers
attending the Qt Developer Days conference. Therefore, I could expect
a good technical knowledge and high-end mobile devices from the target
audience.

A native version of the application was built for devices with Qt
support, and the HTML5 application was for all other devices. The main
target devices were iPhone, Android devices, Windows Phone 7 devices,
and iPad. In addition to these, the application was tested on devices
running Symbian and Meego, as well as desktop browsers.

The conference was expected to have some thousands of attendees, of
which a few hundred was expected to use the web application. In
conferences of this size, network connectivity and reliability is
often a problem. Also, mobile networks other than the \abbr{WiFi}
network supplied by the conference might be too expensive for users
that come from other countries. This is why offline support was
needed.

The client wanted high interactivity and personalization in the
application. Users could save interesting sessions to their favorites,
which were shown in the home view of the application. The home view
was expected to be contextual in taking into account the current time
and showing the ongoing sessions and the remaining time for them, as
well as the time left for later favorite sessions to start. Current
time was also expected to be indicated in the agenda view, where a red
line was to be drawn to the current time for easily visualizing the
ongoing sessions. By default, the agenda view should show the ongoing
day of the conference if possible.

The user interface was required to use the touch input interactions
for panning and zooming the floor maps of the conference venue. Also,
the client wanted a touchable star rating widget on the feedback form
for easy and visual session rating on touch screens.

To sum up, the main requirements are as follows:

\begin{enumerate}
\item Cross-platform support for high-end touch screen smartphones and
  tablets
\item Flexible design and layout to take the available screen into
  account
\item Personalized (session favorites) and contextual experience
  taking the current date and time into account
\item Offline support
\item Floor map view with pinch-to-zoom gesture and panning support
\item Rich agenda and track view with sessions shown visually in a
  timeline
\end{enumerate}

\section{JSONCache Requirements}

JSONCache was designed as a JavaScript network utility library for
unreliable mobile networks. It is used in \abbr{Ajax} data requests
with \abbr{JSON} data. The main idea was to avoid refetching data that
was already fetched and to handle network interruptions without the
user noticing anything. The requirements are as follows:
\begin{enumerate}
\item Cache data in the client side (localStorage) to avoid fetching
  data that has already been transferred
\item Attempt to fetch the data multiple times if the requests fail
  for some reason.
\end{enumerate}
