\chapter{Conclusions}
\label{chapter:conclusions}

In this chapter I present the further improvements for our
implementations and the final conclusions of this work.

\section{Further Work}

At the time of implementing the conference application, I used the
presented tools, YSlow and Page Speed, for analyzing the performance
best practices of the application. However, there are other and newer
tools especially targeted for analyzing mobile web application
performance practices. For example, the online version of Page Speed
offers rules for mobile performance
optimization\footnote{\url{http://code.google.com/speed/page-speed/docs/mobile.html}},
which add some mobile-specific optimizations. This service would
probably offer a better alternative for mobile web application
optimization than the versions I used in the performance analysis.

One of the major optimizations I was aware of, but did not do, was
optimizing images. This was a rather large area where I could have had
significant improvements in the application performance. Going
further, combining several images into one large image sprite would
reduce the number of HTTP requests, and using lossless or lossy image
optimization tools, the size of the images could be further reduced.

The conference application targeted the high-end smartphones that the
users were expected to have in a developer conference. However, a
typical mobile web application does not have this advantage in its
target users. Using progressive enhancement techniques, we can start
from the lowest performing devices and build from there up to the
latest and best-performing devices. This also goes with the very idea
of the Web by providing a truly open and universal access to the
applications.

\section{Discussion}

The Web revolutionized the way we communicate, consume and produce
information in ways that could not have been foreseen twenty years
ago. In addition, the mobile revolution has spread the Web from our
home desks to anywhere we are, to be used at any time of the day. The
roots of the Web lie in openness and universal accessibility for
everyone, and today more and more people can afford a device to access
the vast information spread all over the Web.

One crucial factor in the universality is the open standards used for
defining the protocols and APIs of the Web. HTML5 tackles many of the
growing pains of the Web by defining standards to handle all the
devices capable of accessing the Internet. The set of new
specifications or specification drafts is very large, and growing all
the time.

In this work, I introduced the latest specifications and drafts
related to modern web application development. Some of these
specifications already have very good implementations in several
browsers, but some are just very early drafts. I also presented modern
tools and libraries for developing mobile web applications.

Performance is one of the main components of a successful and usable
application. In this work, I took a practical focus on performance
optimization of mobile web applications. I also tackled other problem
ares in developing these HTML5 applications.

I implemented a schedule application for a developer conference and a
utility library for handling unreliable mobile networks. The
conference application was successfully used in two conferences by
hundreds of people, and the received feedback was excellent. I had to
solve a lot of problems and research solutions in areas that were new
to me. I used the latest HTML5 and related APIs in several parts of
the implementation.

The Web is living very interesting times, and universality in
geography and device types is growing. The browser is the culmination
point of all the new development of the Web, and the new open
standards make the browser a powerful platform for a vast array of
different applications.

Keeping the Web open and accessible for everyone is the key for
technological advancement and innovation in the future. The Web is
here to stay, and with the potential of HTML5 and other modern tools,
we can build powerful applications that improve our day-to-day lives,
as well as applications that revolutionize our lives. There is
grandeur in this view, but without idealism and relentless pursue of
universal accessibility, the full potential of the Open Web might
never be reached.
