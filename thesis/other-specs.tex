\chapter{Other Related Specifications}
\label{chapter:other-related-specifications}

There are lots of specifications related to HTML5 that are considered
to be part of the practical view of all the new Web APIs that are
often referred to as 'HTML5'. Some of these originate from the work of
the \abbr{WHATWG} and some from the work of \abbr{W3C}, some have been
part of HTML5 at some point but have been taken out of it into their
own separate specifications, and some are just new specifications for
the Web that relate to what we can do with HTML, CSS, and JavaScript
in modern web applications. In the following sections, I introduce the
specifications that are of interest within the topic of this work.

\section{Cascading Style Sheets}
\label{section:css}

\abbr{CSS} Level 3 specifications introduce lots of new functionality
for web application styling. Well separated layout layer keeps the
document structure clean, and rich styling and effects capabilities
reduce the need for scripting and provide graceful fallback
functionality for older user agents. By letting the browser handle,
for example, rich user interface animation effects allows developers
to easily optimize the responsiveness and performance of their
applications since the browser can use the most efficient techniques
of the platform to handle these effects. Below I list the main
components and specifications of the \abbr{W3C} CSS working
group\footnote{\url{http://www.w3.org/Style/CSS/members.en.php3}}.

\begin{itemize}

\item \textbf{Selectors}

  \abbr{CSS} selectors are patterns that are used to match elements in
  a \abbr{DOM} tree. The patterns can then be used to apply style
  rules to the matched elements. Selectors can also be used in
  JavaScript to select elements for scripting.

  \abbr{CSS3} defines a set of new selectors \cite{CSS3Selectors} for
  powerful matching of elements in complex DOM trees. These selectors
  are useful for rich interactive web applications and reduce the need
  for scripting for element matching. Efficient selectors are also
  important for performance optimization.

\item \textbf{Transforms}

  2D and 3D transforms \cite{CSStransforms} allow elements to be
  transformed in two-dimensional or three-dimensional space. Elements
  can be translated, rotated, and scaled in their coordinate
  space. The specification defines several 2D and 3D transformation
  functions, which can be used in the transforms.

  Transforms can provide subtle but important user interface effects,
  or they can be used for advanced interactive graphics. Combined
  with, for example, timed animations or transitions, rich user
  interfaces can be built with declarative CSS rules.

\item \textbf{Transitions}

  CSS transitions \cite{CSStransitions} allow element styles to change
  smoothly over a specified duration, and they can be used for simple
  animations. Normally when the value of a CSS property changes, the
  result is seen instantly, but with transitions, the changes can be
  timed and configured for presentational effects.

\item \textbf{Animations}

  Simple animations between two layout states can be done with
  transitions, but for more complex series of changes, CSS animations
  \cite{CSSanimations} can be used.

  The animations specification defines so-called keyframes, which can
  be used to specify the progress of the animation between the start
  and the end states. Animations can also be configured to repeat a
  certain number of times, to alternate between the begin and end
  values, to control the running and paused states, and to delay the
  start time. \cite{CSSanimations}

\item \textbf{Media Queries}

  Older versions of HTML and \abbr{CSS} have already supported targeting
  stylesheets and rules to certain media types like 'screen', 'print',
  or 'mobile'. Media queries expand on this technique by adding extra
  feature queries that can be used to apply styles for certain devices
  and screen sizes. \cite{mediaqueries}

  Media queries can be used to detect the device screen size and
  dimensions, orientation, aspect ratio, color depth,
  etc. \cite{mediaqueries}. Detecting these media features is especially
  useful when using the same HTML markup for different device types. For
  example, a layout of a web application might use more horizontal space
  when used on a desktop browser, but on a mobile device the fragments
  of the layout might be stacked vertically to avoid horizontal
  scrolling. Also background images might be swapped into smaller ones
  with smaller screens.

\item \textbf{Web Fonts}

  Typography is an essential part of design. Traditionally web designers
  have been constrained into only a few ``web safe'' fonts that are
  known to be widely supported between different browsers and platforms.

  CSS3 defines techniques to dynamically load custom fonts and specify
  their properties \cite{cssfonts}. However, different browsers still
  use different formats and developers might have to provide the custom
  font files in all the formats that they want to support.

\end{itemize}

\section{WebGL and Typed Arrays}

\abbr{WebGL} is a low-level 3D rendering API derived from the
\abbr{OpenGL® ES} 2.0 \cite{OpenGL} and it is designed as a rendering
context for the \texttt{canvas} element introduced in HTML5. An
interactive 3D graphics API in the browser is essential for game
development and creates a global platform for cross-platform
games. WebGL was originally developed by the Khronos
Group\footnote{\url{http://www.khronos.org/}} and later the
specification work was participated by browser vendors and 3D
developers. \cite{WebGL}

Being based on the OpenGL ES API, WebGL can run on many different
devices, such as desktop computers, mobile phones, tablets, and
TVs. The use is not constrained into games only; WebGL can also be
used in 3D modeling and design tools, simulations, data visualization,
or in interactive art.

Originated from the WebGL specification, typed arrays have been
separated into their own specification \cite{TypedArrays}. Traditional
JavaScript arrays are not typed, but due to performance reasons, WebGL
API needed more efficient data structures for 3D graphics. These typed
arrays can also be used in non-graphics related contexts, for example,
where efficient processing is needed for large amounts of binary data.

\section{Touch Events}

User interface events have traditionally relied on the input device
being a pointer or a keyboard. Modern mobile devices and tablets,
however, usually have a touch screen. Pointer and keyboard events have
been mapped into the touch interaction, but proper touch events are
needed to support the rich interaction of touch input.

The Touch Events specification \cite{touchevents} defines a set of
events for one or more points of contact on a touch surface. The
specification defines \texttt{touchstart}, \texttt{touchmove},
\texttt{touchend}, and \texttt{touchcancel} events and several new
attributes for the event object. These enable developers to add
functionality for rich interaction such as swipes and multi-touch
gestures.

\section{Files}
\label{section:fileapi}

Local file system access is essential for native
applications. JavaScript storage and database APIs can be used for
certain structured data for caching and other purposes, but are
cumbersome, for example, for large binary files of arbitrary
format. File API specifications \cite{FileAPI, FileAPIWriter,
  FileAPIDir} define interfaces for creating, reading, writing, and
manipulating local files and directories. Error handling and security
sandboxing are also specified in the APIs.

Two versions of the file handling APIs are defined: an asynchronous
API for normal file handling in the main thread and a synchronous API
for file handling in Worker threads (See
Section~\ref{section:webworkers}). Text and binary files can be
manipulated in memory or as \abbr{Blob} \abbr{URLs} with the
\abbr{DOM} API. These capabilities enable web applications to better
optimize network transfer and offline support with large
files. Combined with the Drag and Drop API (See
Section~\ref{section:dragdrop}) and Web Workers (See
Section~\ref{section:webworkers}), HTML5 forms can be greatly enhanced
and optimized with richer interactivity and better performance.

\section{Web Real-time Communication}

Support for different multimedia such as audio and video playing is a
crucial first step into rich and interactive web
applications. However, simply being able to play a video or an audio
file within an HTML document is not enough for multimedia-rich
applications.

Real-time media streaming and playing has been traditionally
implemented with Adobe
Flash\footnote{\url{http://www.adobe.com/products/flashplayer.html}}
and \abbr{RTMP}\footnote{\url{http://www.adobe.com/devnet/rtmp.html}},
but the Web RTC API \cite{WebRTC} brings the ability to do native
media streaming within HTML documents. Also a direct peer-to-peer
streaming communication channel between two user agents is defined.

Combined with the getusermedia API \cite{getusermedia} (see
Section~\ref{section:getusermedia}), streams can be shown or recorded
also from a local media source, such as a web camera. These
specifications have lots of security and privacy issues to handle, but
they promise very strong multimedia capabilities for web applications.

\section{Web Sockets}

Web Sockets API \cite{WebSockets, WebSocketProtocol} defines a two-way
communication protocol for real-time applications between a client,
such as web browser, and a remote server. Because HTTP is a stateless
protocol, highly interactive applications have introduced many
problems when attempting to keep response times and latency
low. Real-time applications such as chat clients have been forced to
rely on complex workarounds to overcome latency issues.

The Web Socket connection can be open or secure, like \abbr{HTTP} and
\abbr{HTTPS}. The API uses a single \abbr{TCP} connection that is kept
open and allows for traffic in both ways. \cite{WebSockets,
  WebSocketProtocol}

The communication protocol specification defines a layer on top of
\abbr{TCP} that defines the connection handshaking with HTTP, an
``origin''-based security model, addressing and protocol naming
mechanism for multiple services on one port and multiple host names on
a single \abbr{IP} address, mechanism to overcome TCP packet length
limits, and a closing handshake to help deal with proxies and other
intermediaries. The intent of Web Sockets is to provide a simple
protocol that works well with HTTP and the existing HTTP
infrastructure, and that is as close to TCP as possible taking
possible security issues into account. \cite{WebSocketProtocol}

\section{Server-Sent Events}

Server-Sent Events specification \cite{ServerSentEvents} defines a
data stream format \texttt{text/event-stream} that can be used to
connect an event listener in the client side to listen for events
initiated by the server. These streams can be used, for example, for
real-time push notifications for data content updates.

The stream data format is very simple, and the API lets the browser
handle the message passing. This helps developers to avoid, for
example, polling a server for updates, which consumes a lot of
computing and networking resources.

\section{Web Workers}
\label{section:webworkers}

The whole \abbr{IT} industry has had a dramatic shift into parallel
computing in recent years. Multicore processors have appeared even in
mobile devices, and the number of cores is increasing in modern
\abbr{CPUs}. This parallellization of processing poses challenges and
opportunities also for application developers. \cite{asanovic2009view}

Following the trend of the computing industry, together with the
proliferation of web technologies, web applications are becoming more
capable and processing-intensive. Introducing a traditional threading
model and making browser APIs (such as \abbr{DOM}) thread-safe would
be an overkill solution and a mismatch to the simplicity and backward
compatibility requirements of the Web.

JavaScript is by design single-threaded with an asynchronous event
model. The whole user interface also runs in the same thread, and
therefore long running JavaScript code freezes the whole interface
during processing. The traditional approach has been to split the code
into small enough pieces, and let the browser handle the scheduling of
user interface events and JavaScript processing. This programming
model is very hard, and combined with unpredictable browser garbage
collection, user interface freezing might be hard to
avoid. \cite{souders2009even}

Web Workers are the proposed solution for parallel computing in
JavaScript. They introduce a simple interface for sandboxed components
with restricted access to browser APIs and an asynchronous
communication channel between the worker and the main
application. \cite{WebWorkers}

Web Workers are external JavaScript files initialized from a web
site. The worker runs in a separate \abbr{OS}-level thread and does
not affect the main user interface thread apart from the communication
channel. The simple API is easy to work with and enables the
long-needed ability of parallel computation in web applications.

\section{Analytics and Timing}

Proper analytics is crucial for performance research and
optimization. JavaScript timers have traditionally been used for
timing and profiling client side interactions, but they can only
provide crude analytics of what happens after the browser has started
to parse the document and executes the JavaScript timing code.

The Navigation Timing specification \cite{NavigationTiming} defines
accurate analytics of events that happens since the user starts to
navigate to the target page and until the page is fully loaded. These
numbers are much more relevant than timing code supplied by the page
itself, since they accurately match the user perceived load time that
usually starts already much earlier than what the target page can
time, for example, when a link is clicked on another page.

Timing APIs are also defined for resources \cite{ResourceTiming} and
general purpose user action profiling \cite{UserTiming}. The
Performance Timeline specification \cite{PerformanceTimeline} defines
a unifying interface to access these performance metrics.

\section{Page Visibility and Timer Control}

Avoiding unneeded work whenever possible is an important performance
optimization concept. Modern browsers usually have a tabbed interface
with possibly dozens of web pages open at a time. In addition, many
interface functionalities use animations and effects to enhance the
user experience, but if the page is not visible, these effects have
little value and might use the \abbr{CPU} time in vain. Moreover,
polling real-time data can use a lot of computing resources even if
the data is not visible to the user.

The Page Visibility specification \cite{PageVisibility} defines an API
for JavaScript to know whether the document is currently visible and a
\texttt{visibilitychange} event to get notified when the document
visibility changes. In addition, the Timing control for script-based
animations specification \cite{TimingControl} and the Efficient Script
Yielding specification \cite{ScriptYielding} define means for
requesting the browser to manage the event queue for efficient
animation and interface event handling. For example, with the
\texttt{requestAnimationFrame} function, the developer can
periodically request for execution time for smooth animations, and
when the page is not visible to the user, the browser can throttle the
animation frame frequency for not using \abbr{CPU} power when it might
be needed more on another page.

\section{Cross-Origin Resource Sharing}

Same-origin policy is an important security restriction in web
browsers preventing web applications to obtain data from other origins
\cite{CORS}. However, mashups and applications using data from
external data APIs have been very popular in recent years, and
techniques have been developed to overcome the restrictions. These
techniques have been unsafe and very limited, so a need for
cross-domain data sharing has risen.

The Cross-Origin Resource Sharing (\abbr{CORS}) specification
\cite{CORS} extends the same-origin policy by allowing the HTTP layer
to indicate allowed origins for data transfer between separate
origins. The specification defines new HTTP headers for indicating
allowed origins and for negotiating access control restrictions with
preflight requests.

\abbr{CORS} is an important specification for modern web applications,
since data is very often distributed to different domains, for
example, when using a content delivery network. It also makes mashup
development with external data easier and safer.

Cross-domain resource usage can also be controlled with the means
defined in the Content Security Policy specification
\cite{ContentSecurityPolicy} and The From-Origin Header specification
\cite{FromOriginHeader}. The specifications define HTTP headers that
can be used to restrict certain content types to be allowed only from
the given origins. In addition to these specifications, the HTML5 Web
Messaging specification \cite{WebMessaging} defines an API for
cross-document messaging.

\section{Device APIs}

One of the main differences between native and web applications is the
device sensor access. In the following sections I present some
specifications that define \abbr{API}s to access the device features
from JavaScript.

\subsection{Geolocation}

Context is one of the main components of a personalized application
\cite{fling2009mobile}. Probably the most important aspect affecting
the context is the physical location of the user. However, only very
crude and error-prone \abbr{IP}-based detection techniques have been
available for web sites.

The Geolocation API defines a standardized JavaScript API for web
applications to query the location of the user. The API is agnostic of
the underlying location information sources; the source might usually
be the \abbr{GPS} chip on a mobile device or a \abbr{WiFi} network
with a known location. The coordinates and their accuracy are provided
by the API. \cite{geolocationAPI}

The implementations usually provide some sort of privacy protection by
asking the user whether they want to grant the application access to
the physical location of the user.

\subsection{Device Orientation}

Knowledge of the physical orientation of a device and changes in the
orientation enable, for example, highly interactive games with rich
input mechanisms. Typical sensor sources for the orientation include
gyroscopes, compasses, and accelerometers \cite{DeviceOrientation}.

The Device Orientation specification \cite{DeviceOrientation} defines
APIs and events to access this sensor data through an abstraction
layer. The data can be used for rich interaction and gestures for a
wide array of applications.

\subsection{User Media}
\label{section:getusermedia}

The getusermedia specification \cite{getusermedia} defines means for
accessing multimedia streams from local devices. The streams can be
audio, video, or both, and the source device can be, for example, the
web camera on a desktop computer or the microphone on a mobile device.

The \texttt{getUserMedia} function can be used to request access to a
multimedia stream, and the stream can then be handled, for example,
with the File API (see Section~\ref{section:fileapi}) for recording to
a file, or used as a source in an \texttt{audio} or a \texttt{video}
element.

\section{Other}

There are also lots of other specifications, drafts, and proposals for
other device feature and sensor data access. Below I list some of
these.

\begin{itemize}

\item \textbf{Web Notifications} \cite{WebNotifications} for
  displaying simple notification to the user

\item \textbf{Fullscreen} \cite{Fullscreen} for controlling fullscreen
  display of the application

\item \textbf{Pointer Lock} \cite{PointerLock} for controlling input
  pointers

\item \textbf{Clipboard} \cite{ClipboardAPI} for using copy, cut, and
  paste operations

\item \textbf{Gamepad} \cite{Gamepad} for a low-level interface to
  gamepad devices

\item \textbf{Battery Status} \cite{BatteryStatusAPI} for accessing
  information about the battery status of the device

\item \textbf{Vibration API} \cite{VibrationAPI} for accessing the
  vibration mechanism of the device

\item \textbf{Network Information} \cite{NetworkInformationAPI} for an
  interface to access the underlying connection information or the
  device

\item \textbf{Contacts} \cite{ContactsAPI} for an interface to access
  the address book of the user

\item \textbf{Web Intents}\footnote{\url{http://webintents.org/}} for
  service discovery and inter-application communication.

\end{itemize}
