\documentclass[a4paper,12pt]{article}

\usepackage[utf8]{inputenc}
\usepackage{hyperref}

\hypersetup{
  pdftitle={Thesis plan - Kimmo Puputti},
  pdfauthor={Kimmo Puputti},
  pdfsubject={Thesis plan},
  pdfkeywords={},
  pdfcreator={pdflatex},
  pdfproducer={pdflatex},
  colorlinks=true,
  linkcolor=blue,
  citecolor=green,
  filecolor=magenta,
  urlcolor=blue
}

\title{Thesis plan}
\author{Kimmo Puputti\\firstname.lastname@tkk.fi\\\url{http://kpuputti.fi/}}

\begin{document}

\maketitle
\thispagestyle{empty}
\setcounter{page}{0}
\clearpage

\setcounter{section}{-1}
\section{Introduction}

The topic of the thesis will be HTML5 as an application platform for
mobile and other devices. HTML5 and related APIs and modern browser
functionality will be assessed as a cross-platform solution compared
to developing applications with each platform's native tools.

Smartphones have become a commodity in the last few years and
therefore several platforms have gained significant share of the
mobile phone market. A cross-platform solution is needed to reach all
potential users of a service, and building a native application for
each platform is very expensive. Furthermore, new platforms and form
factors such as tablets or TVs also become possible targets of the
service.

The only common factor between different devices and platforms is the
browser. Technologies used in web applications are well known and
there are lots of developers around the world who are familiar in
those technologies. Also, new browser APIs are being deployed to help
build personal and contextual applications that match native ones in
functionality, user experience, and performance.

Still, these browser based APIs and technologies are still somewhat
new, and performance and technology support varies between platforms
and devices. The thesis will investigate the available APIs and their
performance in modern smartphone platforms. The work will investigate
especially the performance of the technologies compared to native
applications and also touch the subject of hybrid applications with
some parts built with web technology and others with native
technologies.

The goal is to define areas where web technologies perform well and
areas where native code is needed. Differences between platforms are
assessed keeping in mind the big promises of HTML5 as a
cost-efficient, cross-platform solution for modern applications.

The thesis will also present a modern architecture and useful tools
for mobile web applications. Design considerations and possible
compromises are also investigated in addition to best practices for
high quality applications that can handle slow and flaky mobile
networks and expensive data transfer rates.

\section{Example application}

Qt Developer Days mobile web application (
\url{http://m.qtdevdays2011.qt.nokia.com/} ) is used as an example
application. The app was developed by the author and it was used in
two conferences: Dev Days Munich and Dev Days San Francisco.

The application is built specifically for modern smartphones and
tablets and it uses several of the latest HTML5 APIs.

\section{Example library}

JSONCache ( \url{http://kpuputti.github.com/JSONCache/} ) is a
Javascript library to help data transfer in bad networks. The library
uses HTML5 APIs to cache data and tries to download the data using
several attempts to help dealing with short interruptions that occur
often in mobile networks. The library was also developed by the
author.

\section{Work plan}

The practical application was done in September/October 2011. Articles
have been gathered in November and December and the writing will be
started as soon as possible. January and February are used for
full-time writing work. The goal is to finish writing the thesis in
February.

Experienced tutoring and support is provided by experts at
Futurice. Prof. Petri Vuorimaa from Aalto University will be the
supervisor or the work.

\end{document}
